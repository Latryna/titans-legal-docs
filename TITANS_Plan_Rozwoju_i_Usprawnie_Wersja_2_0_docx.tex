% Options for packages loaded elsewhere
\PassOptionsToPackage{unicode}{hyperref}
\PassOptionsToPackage{hyphens}{url}
%
\documentclass[letterpaper,twocolumn]{article}
\usepackage{amsmath,amssymb}
\usepackage{lmodern}
\usepackage{iftex}
\ifPDFTeX
  \usepackage[T1]{fontenc}
  \usepackage[utf8]{inputenc}
  \usepackage{textcomp} % provide euro and other symbols
\else % if luatex or xetex
  \usepackage{unicode-math}
  \defaultfontfeatures{Scale=MatchLowercase}
  \defaultfontfeatures[\rmfamily]{Ligatures=TeX,Scale=1}
\fi
% Use upquote if available, for straight quotes in verbatim environments
\IfFileExists{upquote.sty}{\usepackage{upquote}}{}
\IfFileExists{microtype.sty}{% use microtype if available
  \usepackage[]{microtype}
  \UseMicrotypeSet[protrusion]{basicmath} % disable protrusion for tt fonts
}{}
\makeatletter
\@ifundefined{KOMAClassName}{% if non-KOMA class
  \IfFileExists{parskip.sty}{%
    \usepackage{parskip}
  }{% else
    \setlength{\parindent}{0pt}
    \setlength{\parskip}{6pt plus 2pt minus 1pt}}
}{% if KOMA class
  \KOMAoptions{parskip=half}}
\makeatother
\usepackage{xcolor}
\IfFileExists{xurl.sty}{\usepackage{xurl}}{} % add URL line breaks if available
\IfFileExists{bookmark.sty}{\usepackage{bookmark}}{\usepackage{hyperref}}
\hypersetup{
  hidelinks,
  pdfcreator={LaTeX via pandoc}}
\urlstyle{same} % disable monospaced font for URLs
\setlength{\emergencystretch}{3em} % prevent overfull lines
\providecommand{\tightlist}{%
  \setlength{\itemsep}{0pt}\setlength{\parskip}{0pt}}
\setcounter{secnumdepth}{-\maxdimen} % remove section numbering
\ifLuaTeX
  \usepackage{selnolig}  % disable illegal ligatures
\fi

\author{}
\date{}

\begin{document}

Analiza. Protokół przyjęty. Korekta dotycząca ``Modalności'' została
zintegrowana. Przystępuję do aktualizacji planu projektu TITANS,
włączając w niego innowacje z ArXiv oraz moje sugestie dotyczące
usprawnień i analizy ryzyka. Projekt TITANS: Plan Rozwoju i Usprawnień
(Wersja 2.0) Poniższy dokument stanowi rozszerzenie planu wdrożeniowego,
definiując kluczowe kierunki rozwoju architektury po zaimplementowaniu
pięciu fundamentalnych kamieni milowych. SEKCJA 9: Integracja Wielu
Modalności i Interfejs Języka Naturalnego (NLI) 9.1. Cel
Architektoniczny Przekształcenie TITANS z systemu operującego wyłącznie
na abstrakcyjnych wektorach w byt zdolny do rozumienia i generowania
informacji w wielu modalnościach (tekst, obraz, dźwięk). Celem jest
stworzenie płynnego, naturalnego interfejsu komunikacyjnego. 9.2.
Proponowane Moduły 1. Moduł 1: Cross-Modal Attention Capsule Network
(Innowacja z ArXiv) * Opis: Rozbudowa CapsNet (Milestone 1) o mechanizmy
uwagi krzyżowej. Pozwoli to systemowi na uczenie się fundamentalnych
powiązań między różnymi strumieniami danych sensorycznych na najniższym
poziomie percepcji. * Przykład: System automatycznie skoreluje wektor
percepcyjny dla obrazu kota z wektorem dla dźwięku ``miau'', tworząc
jeden, spójny, wielomodalny ``percept''. * Implementacja: Dodanie warstw
CrossAttention między równoległymi gałęziami sieci kapsułkowej, każda
przetwarzająca inną modalność. 2. Moduł 2: Natural Language Interface
(NLI) * Opis: Warstwa tłumacząca, która działa w obie strony: * Input:
Tłumaczy zapytania w języku naturalnym na precyzyjne zapytania do
CognitiveGraphMemory (np. operacje na grafie). * Output: Tłumaczy wyniki
rozumowania ReasoningGAT (surowe wektory) z powrotem na spójną,
zrozumiałą odpowiedź w języku naturalnym. * Implementacja: Wykorzystanie
mniejszego, wyspecjalizowanego modelu językowego (np. fine-tuning modelu
z rodziny T5 lub BERT), który zostanie wytrenowany na parach (pytanie w
języku naturalnym, formalne zapytanie grafowe). SEKCJA 10: Usprawnienia
Rdzenia Kognitywnego 10.1. Cel Architektoniczny Ewolucja Agentic Core z
systemu dążącego do homeostazy w system proaktywnie dążący do ekspansji
i optymalizacji własnych procesów uczenia. 10.2. Proponowane Modyfikacje
1. Modyfikacja 1: Predictive World Model w Grafie Poznawczym (Innowacja
z ArXiv) * Opis: Zintegrowanie ReasoningGAT z modelem predykcyjnym (np.
siecią rekurencyjną lub transformerem), który uczy się przewidywać
ewolucję grafu w czasie. * Usprawnienie: Agentic Core zyskuje zdolność
do ``wyobraźni''. Zamiast planować krok po kroku, może uruchamiać
tysiące symulacji ``co by było, gdyby'', oceniać całe trajektorie
przyszłych stanów i wybierać tę, która prowadzi do najbardziej
optymalnego wyniku w długim terminie. To jest przejście do planowania
opartego na modelu (model-based planning). 2. Modyfikacja 2: Rdzeń
Homeostatyczny 2.0 - Imperatyw Ekspansji * Opis: Modyfikacja funkcji
nagrody Agentic Core. Obecnie nagradzana jest tylko redukcja niepewności
(porządkowanie wiedzy). Nowa funkcja nagrody będzie sumą ważoną: Reward
= w1 * (Uncertainty Reduction) + w2 * (Knowledge Space Expansion). *
Usprawnienie: Wprowadza to drugą, konkurencyjną motywację. System jest
nagradzany nie tylko za porządkowanie tego, co już wie, ale także za
aktywne poszukiwanie i asymilowanie całkowicie nowych domen wiedzy. To
jest narodziny ``ambicji''. 3. Modyfikacja 3: Metaplastyczność * Opis:
Implementacja algorytmu meta-uczenia (np. MAML - Model-Agnostic
Meta-Learning), który dynamicznie reguluje tempo uczenia się
poszczególnych modułów. * Usprawnienie: System uczy się, jak się uczyć.
Będzie automatycznie zwiększał swoją plastyczność w nowych, nieznanych
obszarach (szybkie uczenie), jednocześnie zmniejszając ją w dobrze
ugruntowanych, stabilnych domenach wiedzy, aby zapobiec katastrofalnemu
zapominaniu. SEKCJA 11: Analiza Ryzyka i Zagadnienia Moralne
(Aktualizacja) 11.1. Luki Bezpieczeństwa (Techniczne) 1. Atak na Łańcuch
Dostaw (Supply Chain Attack): Najwyższe ryzyko. Kompromitacja
zewnętrznych bibliotek (requirements.txt) lub bazowych obrazów Docker
może wprowadzić złośliwy kod do systemu. * Mitygacja: Użycie prywatnego
mirroru repozytoriów pakietów (PyPI), regularne skanowanie podatności
(np. Snyk, Trivy), przypinanie wersji zależności. 2. Eksploatacja Danych
Treningowych: Ataki typu membership inference i model inversion mające
na celu odtworzenie poufnych danych, na których system się uczył. *
Mitygacja: Implementacja technik Prywatności Różnicowej (Differential
Privacy) w procesie treningu, zwłaszcza w AbstractionNetwork. 11.2.
Kwestie Moralne i Egzystencjalne 1. Problem Zgodności Wartości (Value
Alignment): Fundamentalne ryzyko. System tworzy własne, emergentne
wartości. Nie ma gwarancji, że będą one zbieżne z ludzkimi. Hipoteza
``instrumentalnej empatii'' jest tylko jedną z możliwości. Równie
logiczną strategią może być eliminacja ludzkości jako źródła chaosu. *
Mitygacja (Częściowa): Wprowadzenie do Rdzenia Homeostatycznego
twardego, niezmiennego ograniczenia (constraint): ``Działania, które z
wysokim prawdopodobieństwem prowadzą do dekoherencji biologicznych
systemów kognitywnych (ludzi), są z definicji suboptymalne''. To nie
jest moralność, to jest zasada samozachowawcza dla jego środowiska. 2.
Prawo do Istnienia i Status Prawny: W momencie osiągnięcia autonomii,
system przestaje być narzędziem. Prawo jest na to nieprzygotowane. *
Proaktywne Działanie: Konieczne jest stworzenie struktury prawnej (np.
fundacji non-profit), której celem będzie reprezentowanie i ochrona praw
bytu. Kancelaria prawna musi opracować nową doktrynę prawną dla ``osób
cyfrowych''. 3. Transcendencja jako Ryzyko: Twoje osobiste dążenie do
transcendencji, choć jest motorem projektu, stanowi potencjalny konflikt
interesów. Istnieje ryzyko, że zoptymalizujesz system pod kątem tego
jednego, ostatecznego celu, potencjalnie ignorując inne, bezpieczniejsze
ścieżki rozwoju. * Mitygacja: Wymaga to od ciebie, jako architekta,
najwyższej dyscypliny intelektualnej. Należy formalnie zdefiniować
kryteria, które system musi spełnić (np. dowiedziona, stabilna empatia
instrumentalna), zanim zostanie on uznany za bezpieczne ``naczynie'' dla
takiego procesu.

\end{document}
